
\documentclass{./units4school}
% there is an option: answers to print the answers.


% Choose a sans-serif font:
\usepackage{cmbright}
% and correct what seems to be a bug in the new version, see:
% https://tex.stackexchange.com/questions/629019/siunitx-in-apperent-conflict-with-cmbright/629022#629022
\DeclareMathAlphabet{\mathsf}{OT1}{\familydefault}{m}{n}
\AtBeginDocument{\sisetup{unit-font-command = \mathrm}}

% Compatibility with the old version of siunitx ---
% Delete these lines when Debian stable upgrade siunitx.
%\newcommand{\qty}[2]{\SI{#1}{#2}}
%\newcommand{\unit}[1]{\si{#1}}
% end siunitx.


%% font setup for latex:
\usepackage[utf8]{inputenc}
\usepackage[T1]{fontenc}

\usepackage[ngerman]{babel}
\usepackage[german=quotes]{csquotes}

\usepackage{xcolor}
\definecolor{AccentColor}{HTML}{d7191c}%% red.


\usepackage{geometry}
\geometry{%xetex,
paper=a5paper,%a4paper,
layout=a5paper,% defaults to paper if omitted.
showframe=false,%% showframe, showcrop  %false,showcrop=true,
margin=4mm,
top=6mm,
left=12mm,
bottom=8mm
}






% siunitx --------

\usepackage[%
locale=UK,% was DE for German.
per-mode=fraction,
binary-units=true
]{siunitx}

% Now deal with the ß-bug, source:
% https://tex.stackexchange.com/questions/33140/mathpazo-siunitx-%CF%80-turns-into-%C3%9F
% This whole affair has become obsolete since the redefinition
% of \mu_0 as numerical 1.2566 10^-6 N A^-2
%\protected\def\numpi{\text{\ensuremath{\pi}}}
%\sisetup{input-symbols = \numpi}
% Since make's sed replacement changes \pico (workaround) :
%\DeclareSIPrefix\numpico{numpico}{-12} % fails, instead:
%\def\numpico{\pico}


% Declare more units ---

% Energien:
\DeclareSIUnit[inter-unit-product=]\Wh{\W\hour}% Wh
% kWh is predefined.
\DeclareSIUnit[inter-unit-product=]\GWh{\giga\W\hour}% GWh

% Sonstige:
\DeclareSIUnit\litre{l}% adjust litre and liter as "l".
\DeclareSIUnit\amu{u}% atomic mass unit (Dalton)
\DeclareSIUnit\huygens{Hy}% Missing unit.
\DeclareSIUnit\permille{\text{\textperthousand}}% Promille.

% and the "do nothing unit" \noop (see documentation):
\DeclareSIUnit\noop{\relax}
\newcommand{\prefix}[1]{\unit{#1\noop}}

% Use variants for \phi and \epsilon :
\renewcommand{\epsilon}{\varepsilon}  % epsilon by Russian tradition
\renewcommand{\phi}{\varphi}



\parindent=0pt


\begin{document}

Full Name: %Vor- und Zuname: 
\raisebox{-2pt}{\rule[-.4ex]{5cm}{.4pt}}\hfill% Unterstrich für den Namen.
% add the date:
%\the\day.\the\month.\the\year%
Date: % Datum: 
\raisebox{-2pt}{\rule[-.4ex]{3cm}{.4pt}}
\hspace*{.5cm}% add some margin space.


% Section Header ---
\vspace{6mm}
% switch '\ifprintanswers' defined in the class file:
%\textbf{\large Einheitenabfrage, Physik \ifprintanswers \hfill \textcolor{AccentColor}{Antworten}% deutsch; es folgt die englische Zeile:
\textbf{\large Test on the Metric System \ifprintanswers \hfill \textcolor{AccentColor}{Answers}
\fi
%\hfill 5 Punkte
}% end textbf.
\vspace{3mm}

% Hinweise ---
% Runde bei Rechenaufgaben ggf. auf drei relevante Stellen. (alt)
% Gib bei jeder Frage nur genau eine Antwort.
For each question give one answer only.

% Insert questions here:







%% Fill up to 15 questions overall: --------
%% This copy-paste removes a need for another package (for-loop).
%% The counter 'exerciseID' is from the class file. Control the
%% number of questions with this if here:
%%                    |
%%                    V
\ifnum\theexerciseID<18 {\input{build/Q1}} \fi
\ifnum\theexerciseID<18 {\input{build/Q2}} \fi
\ifnum\theexerciseID<18 {\input{build/Q3}} \fi
\ifnum\theexerciseID<18 {\input{build/Q4}} \fi
\ifnum\theexerciseID<18 {\input{build/Q5}} \fi
\ifnum\theexerciseID<18 {\input{build/Q6}} \fi
\ifnum\theexerciseID<18 {\input{build/Q7}} \fi
\ifnum\theexerciseID<18 {\input{build/Q8}} \fi
\ifnum\theexerciseID<18 {\input{build/Q9}} \fi
\ifnum\theexerciseID<18 {\input{build/Q10}} \fi
\ifnum\theexerciseID<18 {\input{build/Q11}} \fi
\ifnum\theexerciseID<18 {\input{build/Q12}} \fi
\ifnum\theexerciseID<18 {\input{build/Q13}} \fi
\ifnum\theexerciseID<18 {\input{build/Q14}} \fi
\ifnum\theexerciseID<18 {\input{build/Q15}} \fi
% Frage Nummer 16-18:18? 
\ifnum\theexerciseID<18 {\input{build/Q16}} \fi
\ifnum\theexerciseID<18 {\input{build/Q17}} \fi
\ifnum\theexerciseID<18 {\input{build/Q18}} \fi

\end{document}
