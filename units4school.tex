
\documentclass{./units4school}
% there is an option: answers to print the answers.

\usepackage{cmbright}

%% font setup for latex:
\usepackage[utf8]{inputenc}
\usepackage[T1]{fontenc}

\usepackage[ngerman]{babel}
\usepackage[german=quotes]{csquotes}

\usepackage{xcolor}
\definecolor{AccentColor}{HTML}{d7191c}%% red.


\usepackage{geometry}
\geometry{%xetex,
paper=a5paper,%a4paper,
layout=a5paper,% defaults to paper if omitted.
showframe=false,%% showframe, showcrop  %false,showcrop=true,
margin=4mm,
top=6mm,
left=12mm
}






% siunitx --------
\usepackage[locale=DE,per-mode=fraction]{siunitx}
% deal with the ß-bug, source:
% https://tex.stackexchange.com/questions/33140/mathpazo-siunitx-%CF%80-turns-into-%C3%9F
\protected\def\numpi{\text{\ensuremath{\pi}}}
\sisetup{input-symbols = \numpi}
% Since make's sed replacement changes \pico (workaround) :
%\DeclareSIPrefix\numpico{numpico}{-12} % fails, instead:
\def\numpico{\pico}





\parindent=0pt

\AtBeginDocument{
	% add tomorrow's date  
	% error if end of month...
	\advance\day by +1
}


\begin{document}

Name: \rule[-.4ex]{7cm}{.4pt}\hfill%
% add the date:
\the\day.\the\month.\the\year%
\hspace*{.5cm}% add some margin space.


% Section Header ---
\vspace{6mm}
% switch '\ifprintanswers' defined in the class file:
\textbf{\large Einheiten-Abfrage, Physik \ifprintanswers \hfill \textcolor{AccentColor}{Antworten}\fi}
\vspace{3mm}

% Hinweis ---
Runde bei Rechenaufgaben ggf. auf drei relevante Stellen.


%% Fill up to 15 questions overall: --------
%% This copy-paste removes a need for another package (for-loop).
%% The counter 'exerciseID' is from the class file :
\ifnum\theexerciseID<15 {\input{build/Q1}} \fi
\ifnum\theexerciseID<15 {\input{build/Q2}} \fi
\ifnum\theexerciseID<15 {\input{build/Q3}} \fi
\ifnum\theexerciseID<15 {\input{build/Q4}} \fi
\ifnum\theexerciseID<15 {\input{build/Q5}} \fi
\ifnum\theexerciseID<15 {\input{build/Q6}} \fi
\ifnum\theexerciseID<15 {\input{build/Q7}} \fi
\ifnum\theexerciseID<15 {\input{build/Q8}} \fi
\ifnum\theexerciseID<15 {\input{build/Q9}} \fi
\ifnum\theexerciseID<15 {\input{build/Q10}} \fi
\ifnum\theexerciseID<15 {\input{build/Q11}} \fi
\ifnum\theexerciseID<15 {\input{build/Q12}} \fi
\ifnum\theexerciseID<15 {\input{build/Q13}} \fi
\ifnum\theexerciseID<15 {\input{build/Q14}} \fi
\ifnum\theexerciseID<15 {\input{build/Q15}} \fi
%
\end{document}